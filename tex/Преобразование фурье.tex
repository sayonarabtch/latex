\documentclass[a4paper,11pt]{book}
\usepackage[utf8]{inputenc}
\usepackage[english,russian]{babel}
\usepackage{indentfirst}
\begin{document}

	Преобразование Фурье — операция, сопоставляющая одной функции вещественной переменной другую функцию вещественной переменной. Эта новая функция описывает коэффициенты («амплитуды») при разложении исходной функции на элементарные составляющие — гармонические колебания с разными частотами (подобно тому, как музыкальный аккорд может быть выражен в виде суммы музыкальных звуков, которые его составляют).

    Преобразование Фурье функции f вещественной переменной является интегральным и задаётся следующей формулой:
	\[\boxed{f(\omega)=\frac{1}{\sqrt{2\pi}}  \int\limits_{-\infty}^\infty f(x) e^{-ix\omega}  \,dx.}  \qquad \]
	
	Разные источники могут давать определения, отличающиеся от приведённого выше выбором коэффициента перед интегралом, а также знака «−» в показателе экспоненты. Но все свойства будут те же, хотя вид некоторых формул может измениться.
	
\end{document}